\documentclass[12pt,a4paper]{article}
\usepackage[utf8]{inputenc}
\usepackage[T1]{fontenc}
\usepackage{amsmath}
\usepackage{amsfonts}
\usepackage{amssymb}
\usepackage{multicol}
\usepackage{qrcode}
\usepackage{lmodern}
\usepackage{colortbl}%permet de griser les cases
\usepackage{tabularx, multirow}
%\usepackage{lscape}
\usepackage{xcolor}
%\usepackage{graphicx}
\usepackage{tikz,tkz-base}
% Fichier de style stage.sty [UTF8]
% Copyleft Laurent Bretonnière, laurent.bretonniere@gmail.com
% Version du 14/03/2015 

%*********************************************************************************************
% Packages
%*********************************************************************************************

\usepackage[utf8]{inputenc}%			encodage du fichier source (Linux)
\usepackage[TS1,T1]{fontenc}%			gestion des accents (pour les pdf)
\usepackage[french]{babel}%				rajouter éventuellement greek, etc.
\frenchbsetup{CompactItemize=false,StandardLists=true}
\usepackage{enumitem}%
\setenumerate[1]{label=\arabic*/}%
\setenumerate[2]{label=\alph*/}%
%\setlist{font=\bfseries,leftmargin=*}%
\setlist{font=\bfseries,leftmargin=*,topsep=1pt,partopsep=1pt,itemsep=2pt,parsep=1pt}%

\usepackage{textcomp}%					caractères additionnels
\usepackage{amsmath,amssymb}%			pour les maths (1)
\usepackage{amsfonts}%					pour les maths (2)
\usepackage{lmodern}%					remplacer éventuellement par txfonts, fourier, etc.

\usepackage{graphicx}%

% cf site web http://www.khirevich.com/latex/microtype/
%\usepackage[babel=true,kerning=true]{microtype}%
\usepackage{microtype}%

\usepackage{dsfont}%					pour les ensembles de nombres N,Z,D,Q,R,C ...
\usepackage{mathrsfs}%					pour les écritures calligraphiques (genre Cf et Df)
\usepackage[np]{numprint}% page 58  	pour afficher les nombres 3 par 3
\usepackage[e]{esvect}% page 60		vecteurs
\usepackage{stmaryrd}%					pour les intervalles entiers et sslash
\usepackage{empheq}% 					encadrer en mode math
\usepackage{xcolor}%					pour gérer les couleurs
\usepackage{soul}% 						pour les fluos	
\usepackage{xspace}%					gestion des espaces
\usepackage{pifont}% 					trèfle, pique, carreau, coeur
\usepackage{eurosym}%					symbole euro
\usepackage{mathabx}%					choix personnel

\usepackage{hyperref}%
\hypersetup{%
colorlinks=true,%
breaklinks=true,%
citecolor=red,%
urlcolor=blue,%
linkcolor=black,% ou blue
bookmarksopen=false,%
pdfcreator=PDFLaTeX,%
pdfproducer=PDFLaTeX,%
pdfmenubar=true,%
pdftoolbar=true,%
pdfauthor={Laurent Bretonnière},%
pdfkeywords={Mathématiques},%
pdfstartview=XYZ}%

%*********************************************************************************************
% Mathématiques (cf chapitre 7 page 56, LaTeX pour le prof de maths, IREM de Lyon)
%*********************************************************************************************

% Fonctions usuelles
\DeclareMathOperator{\cotan}{cotan}%
\DeclareMathOperator{\ch}{ch}%
\DeclareMathOperator{\sh}{sh}%
\DeclareMathOperator{\thyp}{th}%
\renewcommand{\th}{\thyp}%
\DeclareMathOperator{\Arcsin}{Arcsin}%
\DeclareMathOperator{\Arccos}{Arccos}%
\DeclareMathOperator{\Arctan}{Arctan}%
\DeclareMathOperator{\Argsh}{Argsh}%
\DeclareMathOperator{\Argch}{Argch}%
\DeclareMathOperator{\Argth}{Argth}%
\DeclareMathOperator{\pgcd}{pgcd}%
\DeclareMathOperator{\ppcm}{ppcm}%
\DeclareMathOperator{\card}{card}%

% Composée de fonctions
\newcommand{\rond}{\circ}%

% Multiplication
\newcommand{\x}{\times}

% Constantes usuelles
\renewcommand{\i}{\mathrm{i}}%
\newcommand{\e}{\mathrm{e}}%

% Éléments différentiels
\newcommand{\dt}{\,\textrm{d}t}%
\newcommand{\du}{\,\textrm{d}u}%
\newcommand{\dv}{\,\textrm{d}v}%
\newcommand{\dw}{\,\textrm{d}w}%
\newcommand{\dx}{\,\textrm{d}x}%
\newcommand{\dy}{\,\textrm{d}y}%
\newcommand{\dz}{\,\textrm{d}z}%

% Ensembles de nombres
\newcommand{\ensnb}[1]{\ensuremath{\mathbb{#1}}}%
\newcommand{\N}{\ensnb{N}}%
\newcommand{\Z}{\ensnb{Z}}%
\newcommand{\D}{\ensnb{D}}%
\newcommand{\Q}{\ensnb{Q}}%
\newcommand{\R}{\ensnb{R}}%
\newcommand{\Rp}{\R_{+}}%
\newcommand{\Rm}{\R_{-}}%
\newcommand{\mtsmall}{\fontsize{5pt}{5pt}\selectfont}%
\newcommand{\Rpe}{\R_{\mbox{\mtsmall$+$}}^{\mskip0.4mu\ast}}%
\newcommand{\Rme}{\R_{\mbox{\mtsmall$-$}}^{\mskip0.4mu\ast}}%
\newcommand{\Ret}{\R^{\ast}}% \Re pris pour la partie réelle d'un complexe
\newcommand{\Ne}{\N^{\ast}}%
\newcommand{\Ze}{\Z^{\ast}}%
\newcommand{\C}{\ensnb{C}}%
\newcommand{\Ce}{\C^{\ast}}%

% Noms de points en majuscules en romain (et non pas en italiques)
\DeclareMathSymbol{A}{\mathalpha}{operators}{`A}
\DeclareMathSymbol{B}{\mathalpha}{operators}{`B}
\DeclareMathSymbol{C}{\mathalpha}{operators}{`C}
\DeclareMathSymbol{D}{\mathalpha}{operators}{`D}
\DeclareMathSymbol{E}{\mathalpha}{operators}{`E}
\DeclareMathSymbol{F}{\mathalpha}{operators}{`F}
\DeclareMathSymbol{G}{\mathalpha}{operators}{`G}
\DeclareMathSymbol{H}{\mathalpha}{operators}{`H}
\DeclareMathSymbol{I}{\mathalpha}{operators}{`I}
\DeclareMathSymbol{J}{\mathalpha}{operators}{`J}
\DeclareMathSymbol{K}{\mathalpha}{operators}{`K}
\DeclareMathSymbol{L}{\mathalpha}{operators}{`L}
\DeclareMathSymbol{M}{\mathalpha}{operators}{`M}
\DeclareMathSymbol{N}{\mathalpha}{operators}{`N}
\DeclareMathSymbol{O}{\mathalpha}{operators}{`O}
\DeclareMathSymbol{P}{\mathalpha}{operators}{`P}
\DeclareMathSymbol{Q}{\mathalpha}{operators}{`Q}
\DeclareMathSymbol{R}{\mathalpha}{operators}{`R}
\DeclareMathSymbol{S}{\mathalpha}{operators}{`S}
\DeclareMathSymbol{T}{\mathalpha}{operators}{`T}
\DeclareMathSymbol{U}{\mathalpha}{operators}{`U}
\DeclareMathSymbol{V}{\mathalpha}{operators}{`V}
\DeclareMathSymbol{W}{\mathalpha}{operators}{`W}
\DeclareMathSymbol{X}{\mathalpha}{operators}{`X}
\DeclareMathSymbol{Y}{\mathalpha}{operators}{`Y}
\DeclareMathSymbol{Z}{\mathalpha}{operators}{`Z}

% Raccourci displaystyle + hack :-)
\newcommand{\dps}{\displaystyle}%
\newcommand{\dsp}{\displaystyle}%
\newcommand{\disp}{\displaystyle}%
\everymath{\displaystyle}%

% Mots usuels en mode math
\newcommand{\mtext}[1]{\quad\text{#1}\quad}%
\newcommand{\et}{\mtext{et}}%
\newcommand{\ou}{\mtext{ou}}%
\newcommand{\si}{\mtext{si}}%

% Flèches
\newcommand{\tv}{\shortrightarrow}% tend vers
\renewcommand{\to}{\shortrightarrow}% tend vers
\newcommand{\suit}{\hookrightarrow}% X suit la loi...
\newcommand{\dans}{\longrightarrow}% f:\R\dans\R
\newcommand{\donne}{\longmapsto}% f:x\donne 2x+3
\newcommand{\ppv}{\leftarrow}% flèche <-- d'affectation "prend pour valeur"
\newcommand{\ech}{\leftrightarrow}% double flèche <--> : échange/swap

% Vecteurs
% \vv{AB} en utilisant l'extension \usepackage[e]{esvect}

% Norme et valeur absolue
\newcommand{\abs}[1]{\left\lvert#1\right\rvert}%
\newcommand{\norme}[1]{\left\lVert#1\right\rVert}%

% Complexes
\renewcommand{\Re}{\operatorname{Re}}
\renewcommand{\Im}{\operatorname{Im}}
\renewcommand{\bar}{\overline}

% Matrices
\newcommand{\trans}[1]{{\vphantom{#1}}^{\mathit{t}}\!{#1}}%

% Coefficient binomial
\newcommand{\cb}[2]{\binom{#2}{#1}}%

% Matrice augmentée
\makeatletter
\renewcommand*\env@matrix[1][*\c@MaxMatrixCols c]{%
  \hskip -\arraycolsep
  \let\@ifnextchar\new@ifnextchar
  \array{#1}}
\makeatother

% Parallèles et perpendiculaires
\newcommand{\para}{\sslash}%
% perp pour perpendiculaire

% Intervalles
\newcommand{\intervalle}[4]{\mathchoice%
{\left#1#2\mathclose{}\mathpunct{},#3\right#4}% mode \displaystyle
{\mathopen{#1}#2\mathclose{}\mathpunct{},#3\mathclose{#4}}% mode \textstyle
{\mathopen{#1}#2\mathclose{}\mathpunct{},#3\mathclose{#4}}% mode \scriptstyle
{\mathopen{#1}#2\mathclose{}\mathpunct{},#3\mathclose{#4}}% mode \scriptscriptstyle
}%

\newcommand{\intff}[2]{\intervalle{[}{#1}{#2}{]}}%
\newcommand{\intof}[2]{\intervalle{]}{#1}{#2}{]}}%
\newcommand{\intfo}[2]{\intervalle{[}{#1}{#2}{[}}%
\newcommand{\intoo}[2]{\intervalle{]}{#1}{#2}{[}}%

% Ancienne configuration
%\newcommand{\intervalle}[4]{\mathopen{#1}#2\mathclose{}\mathpunct{},#3\mathclose{#4}}%
%\newcommand{\intoo}[2]{\ensuremath{\,\left]  #1 \,, #2  \right[\, }}%
%\newcommand{\intof}[2]{\ensuremath{\,\left]  #1 \,, #2  \right]\, }}%
%\newcommand{\intfo}[2]{\ensuremath{\,\left[  #1 \,, #2  \right[\, }}%
%\newcommand{\intff}[2]{\ensuremath{\,\left[  #1 \,, #2  \right]\, }}%

% Intervalles entiers
\newcommand{\intn}[2]{\intervalle{\llbracket}{#1}{#2}{\rrbracket}}%

% Ensembles et Probabilités
\newcommand{\vide}{\varnothing}% ensemble vide
\newcommand{\union}{\cup}%
\newcommand{\inter}{\cap}%
\newcommand{\Union}{\bigcup}%
\newcommand{\Inter}{\bigcap}%
\newcommand{\compl}{\complement}% complémentaire
\newcommand{\inclus}{\subseteq}% inclus : je n'aime pas \subset je préfère \subseteq...
\newcommand{\inclusstrict}{\subsetneq}% inclus au sens strict ...
\newcommand{\contient}{\supseteq}% contient
\newcommand{\contientstrict}{\supsetneq}% contient au sens strict ...
\newcommand{\prive}{\setminus}% privé de ...
\renewcommand{\P}{\mathrm{P}} % probabilité
\newcommand{\V}{\mathrm{V}} % variance
\newcommand{\E}{\mathrm{E}} % espérance

% ensemble des ... tels que ...
\newcommand{\enstq}[2]{\left\{#1\,\;\middle|\;\,#2\right\}}%

% Pointillés anti-diagonale
\newcommand{\adots}{\mathinner{\mkern2mu\raise 1pt\hbox{.}\mkern3mu\raise 4pt\hbox{.}\mkern1mu\raise 7pt\hbox{.}}}%

% Partie entière
\newcommand{\ent}[1]{\left\lfloor#1\right\rfloor}% partie entière (première notation)
\newcommand{\Ent}[1]{\textrm{Ent}\mathopen{}\left(#1\right)}% partie entière (deuxième notation)

% Angle
\renewcommand{\angle}{\widehat}%

% Limites
\newcommand{\iy}{\infty}% 
\newcommand{\ii}{\infty}%
\newcommand{\zp}{0^{+}}%
\newcommand{\zm}{0^{-}}%

% Encadrement d'une formule
%\begin{empheq}[box=\fbox]{equation*}
% ...    
%\end{empheq}

% Couleurs
% http://www.latextemplates.com/svgnames-colors
\definecolor{bleu1}{HTML}{000080}%
\definecolor{grispale}{RGB}{245 245 245}%
\definecolor{bistre}{rgb}{.75 .50 .30}%
\definecolor{grisclair}{gray}{0.8}%
\definecolor{bleuclair}{rgb}{0.7, 0.7, 1.0}%
\definecolor{rosepale}{rgb}{1.0, 0.7, 1.0}%

% Fluos !
\newcommand{\fluo}[1]{\sethlcolor{rosepale}\hl{#1}}%

% Lettres calligraphiées
\newcommand{\Cf}{\mathscr{C}_f}%
\newcommand{\Df}{\mathscr{D}_f}%
\newcommand{\Cg}{\mathscr{C}_g}%
\newcommand{\Dg}{\mathscr{D}_g}%
\newcommand{\Ch}{\mathscr{C}_h}%
\newcommand{\Dh}{\mathscr{D}_h}%

% Degré
\newcommand{\Degre}{\ensuremath{^\circ}}

% Lettres grecques
\renewcommand{\epsilon}{\varepsilon}%
\renewcommand{\phi}{\varphi}%

% Mots usuels
\newcommand{\ie}{\;\textit{i.e.}\;\xspace}
\newcommand{\cad}{c'est--à--dire\xspace}%
\newcommand{\pourcent}{\unskip~\%\xspace}%
\newcommand{\ssi}{si et seulement si\xspace}%
\newcommand{\eve}{événement\xspace}%
\newcommand{\eves}{événements\xspace}%
\newcommand{\sev}{sous-espace vectoriel\xspace}%
\newcommand{\ipp}{intégration par parties\xspace}%
\newcommand{\iaf}{inégalité des accroissements finis\xspace}%
\newcommand{\tvi}{théorème des valeurs intermédiaires\xspace}%
\newcommand{\fpt}{formule des probabilités totales\xspace}%
\newcommand{\fpc}{formule des probabilités composées\xspace}%
\newcommand{\sce}{système complet d'événements\xspace}%
\newcommand{\srld}{suite récurrence linéaire d'ordre $2$\xspace}%
\newcommand{\sag}{suite arithmético-géométrique\xspace}%

% Guillements français
\newcommand{\guill}[1]{%
\og{}#1\fg{}}%


% Inégalités
\renewcommand{\leq}{\leqslant}%
\renewcommand{\geq}{\geqslant}%
\renewcommand{\le}{\leqslant}%
\renewcommand{\ge}{\geqslant}%
\newcommand{\pg}{\geqslant}%
\newcommand{\pp}{\leqslant}%

% Environ
\newcommand{\environ}{\simeq}%
\renewcommand{\approx}{\simeq}%


% Tableau de variations en TikZ
\usepackage{tikz,tkz-tab}%
\definecolor{fondpaille}{rgb}{1,1,1}%

% Aire d'une figure géométrique
\newcommand{\aire}{\text{aire}}%

% Paramétrage de quelques variables
\setlength{\columnsep}{1cm}%
\setlength{\columnseprule}{0.4pt}%
\setlength{\parindent}{0pt}%

% Mathématiciens
\newcommand{\GJ}{Gauss\,--\,Jordan\xspace}%
\newcommand{\KH}{König\,--\,Huygens\xspace}%

% Siècle en lettres romains
\newcommand{\siecle}[1]{\textsc{\romannumeral #1}\textsuperscript{e}~si\`ecle}%

% Couleurs jeu de carte
\newcommand{\pique}{\ding{171}}%
\newcommand{\coeur}{\ding{170}}%
\newcommand{\carreau}{\ding{169}}%
\newcommand{\trefle}{\ding{168}}%

% Exercices (fiche)
\renewcommand*{\hrulefill}[2][0pt]{\leavevmode \leaders \hbox to 1pt{\rule[#1]{1pt}{#2}} \hfill \kern 0pt}%

\newcounter{numexercice}%

\newenvironment{exercice}{\stepcounter{numexercice}\ovalbox{\textbf{\thenumexercice}}\hrulefill[3pt]{0.5pt}\par\medskip\nopagebreak[4]}{\medskip}

% Trait de la largeur de la feuille
\newcommand{\trait}{\hbox{\raisebox{0.4em}{\vrule depth 0pt height 0.4pt width \textwidth}\linebreak}}%

\newcommand{\demitrait}{\hbox{\raisebox{0.4em}{\vrule depth 0pt height 0.4pt width 0.48\textwidth}\linebreak}}%

\newcommand{\LV}{Lycée Le Verrier, Saint\,--\,Lô}%
%\newcommand{\itb}{\item[\textbullet]}%
\newcommand{\itb}{\item}%
%\newcommand{\Gaffe}{\ding{54}\ding{54}\ding{54}\quad}%
\newcommand{\gaffe}{\ding{56}\ding{56}\ding{56}\quad}%


% Fichier de style stage2.sty [UTF8]
% Copyleft Laurent Bretonnière, laurent.bretonniere@gmail.com
% Version du 16/03/2015

\usepackage{mathtools}%	
\usepackage{fancybox}%
\usepackage{lastpage}%

\usepackage{fancyhdr}%
\renewcommand{\headrulewidth}{0.8pt}%
\renewcommand{\footrulewidth}{0.8pt}%

\usepackage[tikz]{bclogo}%
\renewcommand\bcStyleTitre[1]{\normalsize\textbf{#1}\smallskip}%
\renewcommand\logowidth{0pt}%

\newcommand{\fin}{\begin{center}%
$\clubsuit\clubsuit\clubsuit$%
\end{center}}%

\newcommand{\un}{\ding{192}\xspace}%
\newcommand{\deux}{\ding{193}\xspace}%
\newcommand{\trois}{\ding{194}\xspace}%
\newcommand{\quatre}{\ding{195}\xspace}%
\newcommand{\cinq}{\ding{196}\xspace}%
\newcommand{\six}{\ding{197}\xspace}%
\newcommand{\sept}{\ding{198}\xspace}%
\newcommand{\huit}{\ding{199}\xspace}%
\newcommand{\neuf}{\ding{200}\xspace}%

\setlength{\headheight}{15pt}%

%*********************************************************************************************
% Cours
%*********************************************************************************************

\usepackage[Lenny]{fncychap}%
\ChNumVar{\fontsize{76}{80}\usefont{OT1}{pzc}{m}{n}\selectfont}%
\ChTitleVar{\raggedleft\Huge\sffamily\bfseries}%

\renewcommand{\thesection}{\Roman{section})}%
\renewcommand{\thesubsection}{\arabic{subsection})}%
\renewcommand{\thesubsubsection}{\alph{subsubsection})}%

%*********************************************************************************************
% Environnements prédéfinis BCLOGO
%*********************************************************************************************

%% Lemme
\newenvironment{lem}{\begin{bclogo}[couleurBord=black!50,arrondi=0.1,logo=\hspace{17pt},barre=none]{Lemme :}}{\end{bclogo}\medskip}%

%% Proposition
\newenvironment{prop}[1][]{\begin{bclogo}[couleurBord=black!50,arrondi=0.1,logo=\hspace{17pt},barre=none]{Proposition :~#1}}{\end{bclogo}\medskip}%

%% Théorème
\newlength{\textlarg}
\settowidth{\textlarg}{~}
\newenvironment{theo}[1][\hspace{-\textlarg} :]{\begin{bclogo}[couleur=black!5,couleurBord=black!50,arrondi=0.1,logo=\hspace{17pt}, barre=none]{Théorème~#1}}{\end{bclogo}\medskip}%

\newenvironment{theon}[1][]{\begin{bclogo}[couleur=black!5,couleurBord=black!50,arrondi=0.1,logo=\hspace{17pt}, barre=none]{Théorème :~#1}}{\end{bclogo}\medskip}%

%% Corollaire
\newenvironment{coro}[1][]{\begin{bclogo}[couleurBord=black!50,arrondi=0.1,logo=\hspace{17pt},barre=none]{Corollaire :~#1}}{\end{bclogo}\medskip}%

%% Définition(s)

\newenvironment{defi}{\begin{bclogo}[couleurBord=black!50,arrondi=0.1,logo=\hspace{17pt}, barre=none]{Définition :}}{\end{bclogo}\medskip}%

\newenvironment{defis}{\begin{bclogo}[couleurBord=black!50,arrondi=0.1,logo=\hspace{17pt}, barre=none]{Définitions :}}{\end{bclogo}\medskip}%

%% Preuve
\newenvironment{pf}{\renewcommand\logowidth{17pt}\begin{bclogo}[noborder=true,logo=\hspace{17pt},couleurBarre=black!25,epBarre=3.5]{Preuve :}}{\hspace*{\fill}$\Box$\end{bclogo}\smallskip\renewcommand\logowidth{0pt}}%

%\blacksquare

%% Notation
\newenvironment{nota}{\begin{bclogo}[couleurBord=black!50,arrondi=0.1,logo=\hspace{17pt},barre=none]{Notation :}}{\medskip}%

%% Exercice et Exercice-type
\newenvironment{exo}{$\circledast$ \quad\textsc{\underline{exercice} :}~}{\hspace*{\fill}$\circledast$\vskip 8pt}
\newenvironment{type}{$\blacktriangleright$ \quad\textsc{exercice-type :}~}{\hspace*{\fill}$\blacktriangleleft$\vskip 8pt}

%% Exemple(s)
\newenvironment{exem}{\textbf{Exemple :}~}{\medskip}
\newenvironment{exems}{\textbf{Exemples :}~}{\medskip}

%% Remarque(s)
\newenvironment{rem}{\textbf{Remarque :}~}{\medskip}
\newenvironment{rems}{\textbf{Remarques :}~}{\medskip}

%% Rappel(s)
\newenvironment{rap}{\textbf{Rappel :}~}{\medskip}
\newenvironment{raps}{\textbf{Rappels :}~}{\medskip}

%% Cas particulier(s)
\newenvironment{cp}{\textbf{Cas particulier :}~}{\medskip}
\newenvironment{cps}{\textbf{Cas particuliers :}~}{\medskip}

%% Application
\newenvironment{appli}{\textbf{Application :}~}{\medskip}%{\medskip} 

%******************************************

%Permet le code python sur lateX
\usepackage{minted}
\usemintedstyle{lovelace}

%box exercice
\usepackage{tcolorbox}
\newtcolorbox{mybox}[1]{colback=yellow!5!,colframe=yellow!50!black,colbacktitle=yellow!75!black,fonttitle=\bfseries,
title=#1}

%%Propriété
\newenvironment{pro}[1][]{\begin{bclogo}[couleurBord=black!50,arrondi=0.1,logo=\hspace{17pt},barre=none]{Propriété :~#1}}{\end{bclogo}\medskip}%


%Permet de mettre les coordonnées d'un vecteur
\newcommand*{\Coord}[3]{% 
  \ensuremath{\overrightarrow{#1}\, 
    \begin{pmatrix} 
      #2\\ 
      #3 
    \end{pmatrix}}}


\usepackage[left=2cm,right=2cm,top=2cm,bottom=2cm]{geometry}
\def\Oij{$\left(\text{O},~\vec{i},~\vec{j}\right)$}
\usepackage{fancyhdr}
\usepackage{MnSymbol,wasysym}

%Permet le code python sur lateX
\usepackage{minted}
\usemintedstyle{lovelace}



\begin{document}
\textbf{2nd} \hfill \textbf{Les vecteurs -2ème partie} \hfill Lycée Jean Rostand\\
\trait 

\section{Coordonnées d'un vecteur}
\begin{defi}
Soit M un point quelconque d'un repère \Oij{} et un vecteur $\vv{u}$ tel que: $\vv{OM}=\vv{u}$.\\
Les \textbf{coordonnées du vecteur $\vv{u}$} sont les coordonnées du point M.\\
On note : $\Coord{u}{x}{y}$
\end{defi}


    


\begin{center}
\begin{tikzpicture}[scale=0.8]
    \tkzInit[xmin=-3, xmax=5, ymin=-2,ymax=5]
    \tkzDrawXY
    \draw[lightgray,very thin] (-3,-2) grid (5,5);
    \draw [->,>=latex] (0,0) -- (1,0);
    \draw [->,>=latex] (0,0) -- (0,1) ;
    \draw (-0.2,0) node [below,] {$O$};
    \draw (0,0.5) node [left] {$\vv{j}$};
      \draw (0.5,0) node [below]{$\vv{i}$};
     \draw [->,>=latex,red] (0,0) -- (4,2);
     \draw [->,>=latex,red] (-2,2) -- (2,4);
     \draw (4,2) node [above] {$M$};
     \draw (2,1.2) node [above,red]{$\vv{OM}$};
     \draw (-0.5,3) node [above,red]{$\vv{u}$};

\end{tikzpicture}
\end{center}



\begin{mybox}{Exercice n°1}
Déterminer les coordonnées des vecteurs $\vv{AB}$ , $\vv{CD}$ et $\vv{EF}$ par lecture graphique.
\end{mybox}



\compo[0.7]{
\begin{center}
\begin{tikzpicture}[scale=0.8]
    \tkzInit[xmin=-4, xmax=6, ymin=-5,ymax=4]
    \tkzDrawXY
    \draw[lightgray,very thin] (-4,-5) grid (6,4);
    \draw [->,>=latex] (0,0) -- (1,0);
    \draw [->,>=latex] (0,0) -- (0,1) ;
    \draw (-0.2,0) node [below,] {$O$};
    \draw (0,0.5) node [left] {$\vv{j}$};
      \draw (0.5,0) node [below]{$\vv{i}$};
     \draw [->,>=latex] (2,1) -- (5,3);
     \draw [->,>=latex] (-1,-2) -- (-2,3);
     \draw [->,>=latex] (1,-4) -- (4,-2);
     \draw (2,1) node [above] {$A$};
     \draw (5,3) node [above] {$B$};
     \draw (-1,-2) node [right] {$C$};
     \draw (-2,3) node [right] {$D$};
     \draw (1,-4) node [above] {$E$};
     \draw (4,-2) node [above] {$F$};
     

\end{tikzpicture}
\end{center}
}{\qrcode[height=2cm]{https://youtu.be/8PyiMHtp1fE}}

\begin{pro}
\compo[0.7]{
Soit $A$ et $B$ deux points de coordonnées $(x_A,y_A)$ et $(x_B,y_B)$ dans un repère \Oij{}.\\
Le vecteur $\vv{AB}$ a pour coordonnées :
$\begin{pmatrix} 
      x_B-x_A\\ 
      y_B-y_A 
    \end{pmatrix}$
    
}{\qrcode[height=2cm]{https://youtu.be/wnNzmod2tMM}}    
\end{pro}

\begin{mybox}{Exercice n°2}
Déterminer par calcul les coordonnées du vecteurs $\vv{AB}$ tel que $A(-2; 1)$ et $B(3; 4)$.
\end{mybox}

\begin{framed}
\vspace{3cm}
\end{framed}

\begin{pro}
Soit $\vv{u}$ et $\vv{v}$ deux vecteurs de coordonnées 
$\begin{pmatrix} 
      x\\ 
      y
\end{pmatrix}$
et 
$\begin{pmatrix} 
      x'\\ 
      y'
\end{pmatrix}$
dans un repère \Oij{} et un réel $k$.
\begin{itemize}
    \item $\vv{u}=\vv{v}$ équivaut à $x=x'$ et $y=y'$
    \item Le vecteur $\vv{u}+\vv{v}$a pour coordonnées 
    $\begin{pmatrix} 
      x+x'\\ 
      y+y'
\end{pmatrix}$
\item Le vecteur $k\vv{u}$ a pour coordonnées
 $\begin{pmatrix} 
      kx'\\ 
      ky'
\end{pmatrix}$
\end{itemize}
\end{pro}

\begin{mybox}{Exercice n°3}
Soient les points $C(5;-1)$ et $D(-2; 4)$ et $\Coord{u}{-1}{6}$ un vecteur du repère.\\
\begin{enumerate}
    \item Calculer les coordonnées de $\vv{CD}$
    \item  Calculer les coordonnées de $\vv{CD}+\vv{u}$
    \item Calculer les coordonnées de $3\vv{u}$ et $\vv{DC}$
\end{enumerate}
\end{mybox}


\begin{framed}
\vspace{3cm}
\end{framed}


\begin{mybox}{Exercice n°4}
Dans un repère, soit les points $A(2; 4)$ et $B(-2; 2)$ et $C(4; -2)$ et $D(8;0)$\\

\begin{enumerate}
    \item Placer les points dans un repère \Oij{}. Quel conjecture peut-on faire sur le quadrilatère $AB\textbf{C}D$ ?
    \item Démontrer que $\vv{AD}=\vv{BC}$
    \item Déterminer les coordonnées du point $E$ tel que $ABEC$ soit un parallélogramme.
\end{enumerate}

Déterminer les coordonnées du points $D$ tel que $AB\textbf{C}D$ soit un parallélogramme
\end{mybox}

\begin{framed}
\vspace{3cm}
\end{framed}





\begin{mybox}{Exercice n°5}
Dans un repère, soit les points $A(1; 2)$ et $B(-4; 3)$ et $C(1; 2)$.\\
Déterminer les coordonnées du points $D$ tel que $AB\textbf{C}D$ soit un parallélogramme
\end{mybox}

\begin{framed}
\vspace{3cm}
\end{framed}


\begin{mybox}{Exercice n°6}
Dans un repère \Oij{}, soit les points $A(-2; 1)$ et $K(2; 3)$ et $B(6; 7)$ et $L(1;6)$\\
\begin{enumerate}
    \item Calculer les coordonnées de $\vv{IA}$ et $\vv{JK}$
    \item Déterminer les coordonnées du point $D$ tel que $\vv{KD}=-\vv{BL}$
\end{enumerate}
\end{mybox}

\begin{framed}
\vspace{3cm}
\end{framed}











\section{Distance ou norme d'un vecteur}

\begin{pro}
Soit $A$ et $B$ deux points de coordonnées $(x_A,y_A)$ et $(x_B,y_B)$ dans un repère \Oij{} , alors :
$$AB=\sqrt{(x_B-x_A)^2+(y_B-y_A)^2}$$

La norme du vecteur $\vv{AB}$ est la distance $AB$ c'est-à-dire la longueur sur segment $[AB]$ .\\
Elle se note à l'aide d'une double barre : $\norme{\vv{AB}}$ 
\end{pro}


\begin{mybox}{Exercice n°7}
Dans un repère \Oij{}, soit les points $A(3;2)$ et $B(2; -2)$ .

\begin{enumerate}
    \item Calculer les coordonnées du vecteur $\vv{AB}$.
    \item En déduire la norme $\norme{\vv{AB}}$ .
\end{enumerate}

\end{mybox}

\begin{framed}
\vspace{3cm}
\end{framed}


\section{Coordonnées du milieu d'un segment}


\begin{pro}
Soit $A$ et $B$ deux points de coordonnées $(x_A,y_A)$ et $(x_B,y_B)$ dans un repère \Oij{}.
Le milieu $M$ du segment $[AB]$ a pour coordonnées:

$$M\left(\dfrac{x_A+x_B}{2};\dfrac{y_A+y_B}{2}\right)$$
\end{pro}

\begin{mybox}{Exercice n°6}
Dans un repère \Oij{}, soit les points $A(2;3)$ et $B(-2; 1)$ et $C(3;-1)$.\\
Calculer les coordonnées de $M$, $N$ et $P$ milieux respectifs de $[AB]$, $[AC]$ et $[BC]$.
\end{mybox}

\begin{framed}
\vspace{3cm}
\end{framed}


\section{Colinéarité de deux vecteurs}

\begin{defi}
Deux vecteurs $\vv{u}$ et $\vv{v}$ sont colinéaires lorsqu'ils existent un réel $k$ non nul tel que $\vv{u}=k\x\vv{v}$
\end{defi}


\begin{center}
    

\begin{tikzpicture}[scale=0.8]
    \tkzInit[xmin=-5, xmax=5, ymin=-4,ymax=5]
    \tkzDrawXY
    \draw[lightgray,very thin] (-5,-4) grid (5,5);
    \draw [->,>=latex] (0,0) -- (1,0);
    \draw [->,>=latex] (0,0) -- (0,1) ;
    \draw (-0.2,0) node [below] {$O$};
    \draw (0,0.5) node [left] {$\vv{j}$};
      \draw (0.5,0) node [below]{$\vv{i}$};
     \draw [->,>=latex] (-2,3) -- (-4,1);
     \draw [->,>=latex] (1,2.5) -- (2,3.5);
     \draw [->,>=latex] (1,-3) -- (4,0);
    \draw (-3,2.2) node [left]{$-2\vv{u}$};
    \draw (1.6,3.2) node [left]{$\vv{u}$};
    \draw (2.6,-1.4) node [left]{$3\vv{u}$};
\end{tikzpicture}
\end{center}

\begin{mybox}{Exercice n°7}
Soient $\Coord{u}{3}{-2}$ et $\Coord{v}{-12}{8}$.
Ces deux vecteurs sont-ils colinéaires ?
\end{mybox}

\begin{framed}
\vspace{3cm}
\end{framed}


















\begin{defi}
Soient $\Coord{u}{x}{y}$ et $\Coord{v}{x'}{y'}$ deux vecteurs dans un repère du plan.\\
Le déterminant de $\vv{u}$ et $\vv{v}$, noté $det(\vv{u};\vv{v})$, est le nombre tel que : $$det(\vv{u};\vv{v})=\begin{vmatrix} x & x' \\ y & y' \end{vmatrix}=xy'-x'y$$

\end{defi}


\begin{pro}
$\Coord{u}{x}{y}$ et $\Coord{v}{x'}{y'}$ sont colinéaires si et seulement si $det(\vv{u};\vv{v})=0$
\end{pro}

\begin{pro}
\textbf{Conséquences sur l'alignement et le parallélisme}

\begin{itemize}
    \item $\vv{AB}$ et $\vv{MN}$ sont colinéaires ssi $(AB)\sslash (MN)$
    \item $\vv{AB}$ et $\vv{AC}$ sont colinéaires ssi $A$,$B$ et $C$ sont alignés.
\end{itemize}
\end{pro}

\begin{mybox}{Exercice n°8}
Dans un repère \Oij{}, soit les points $A(1;1)$, $B(5; 3)$, $C(2;3)$, $D(4;4)$ et $E(-1;-3)$\\

\begin{enumerate}
    \item Démontrer que les points $A$, $C$ et $E$ sont alignés.
    \item Démontrer que les droites $(AB)$ et $(CD)$ sont $\sslash$
\end{enumerate}


\end{mybox}

\begin{framed}
\vspace{6cm}
\end{framed}


\begin{mybox}{Exercice n°9}
Dans un repère \Oij{}, soit les points $A(-2;-1)$, $B(2; 1)$ et $M(5;y)$\\

Déterminer $y$ pour que $A$, $B$ et $M$ soient alignés.

\end{mybox}

\begin{framed}
\vspace{3cm}
\end{framed}
\section{Equation cartésienne de droite et vecteur directeur.}
\begin{defi}
Soit $\mathscr{D}$ une droite, et $A$ et $B$ deux points de $\mathscr{D}$.\\
On appelle \textbf{vecteur directeur} de $\mathscr{D}$ tout vecteurs $\vv{u}$ non nul colinéaire à $\vv{AB}$.\\
\textit{\underline {Autrement dit}: le vecteur $\vv{u}$ donne la direction de la droite $\mathscr{D}$}
\end{defi}

\begin{center}
    
\begin{tikzpicture}[scale=0.8]
    \tkzInit[xmin=-3, xmax=6, ymin=-2,ymax=5]
    \tkzDrawXY
    \draw[lightgray,very thin] (-3,-2) grid (6,5);
    \draw  (-2,-2) -- (6,2);
     \draw [->,>=latex] (-2,2) -- (2,4);
      \draw [->,>=latex] (2,2) -- (4,3);
     \draw (2,0) node [above] {$\mathscr{D}$};
     \draw (0,-1) node {$\bullet$};
     \draw (0,-1) node [below] {$A$};
     \draw (4,1) node {$\bullet$};
     \draw (4,1) node [below]{$B$};
     \draw (-0.5,3) node [above]{$\vv{u}$};
     \draw (3,2.5) node [above]{$\vv{v}$};

\end{tikzpicture}
\end{center}

\begin{rem}
Un vecteur directeur n'est pas unique: ici les vecteurs $\vv{u}$ et $\vv{v}$ sont des vecteurs directeurs de la droite $(AB)$.
\end{rem}

\begin{mybox}{Exercice n°10}
Soit \Oij{} un repère du plan.\\
Donner \textbf{des} vecteurs directeurs des droites $\mathscr{D}{_1}$,$\mathscr{D}{_2}$,$\mathscr{D}{_3}$ et $\mathscr{D}{_4}$.
\end{mybox}


\compo[0.7]{
\begin{center}
\begin{tikzpicture}[scale=1]
    \tkzInit[xmin=-3, xmax=6, ymin=-2,ymax=6]
    \tkzDrawXY
    \draw[lightgray,very thin] (-3,-2) grid (6,6);
    \draw [domain=-1.5:2.5,samples=100] plot (\x,{2*\x+1});
    \draw [domain=-3:5,samples=100] plot (\x,{-\x+3});
    \draw [-] (-3,4) -- (6,4);
    \draw [-] (4,-2) -- (4,6);
    \draw (2,5) node[right]{$\mathscr{D}{_1}$};
    \draw (3,0) node[above]{$\mathscr{D}{_3}$};
    \draw (5.5,4) node[above]{$\mathscr{D}{_2}$};
    \draw (4,2) node[right]{$\mathscr{D}{_4}$};
\end{tikzpicture}
\end{center}
}{\qrcode[height=2cm]{https://youtu.be/6VdSz-0QT4Y}}


\begin{pro}
Une équation cartésienne de la droite passant par le point $A(x_A;y_A)$ et de vecteur directeur $\Coord{u}{-b}{a}$ est de la forme $ax+by+c=0$.
\end{pro}

\begin{pf}
Un point $M(x;y)$ appartient à cette droite ssi $\Coord{AM}{x-x_A}{y-y_A}$ et $\Coord{u}{-b}{a}$ sont colinéaires.\\
Il faut donc que $det(\vv{AM};\vv{u})=0$.\\
Il vient, $$det(\vv{AM};\vv{u})=\begin{vmatrix} x-x_A & -b \\ y-y_A & a \end{vmatrix}=a(x-x_A)-(-b)(y-y_A)=0$$

En développant, l'équation peut s'écrire:

$$ax+by+(-ax_A-by_A)=0$$

On pose $c=-ax_A-by_A$

d'où: $$\boxed{ax+by+c=0}$$

\end{pf}



\begin{reci}
Si les coordonnées $(x;y)$ d'un point $M$ vérifient l'équation $ax+by+c=0$, alors $M$ appartient à la droite dont un vecteur directeur est $\Coord{u}{-b}{a}$.


\end{reci}

\begin{mybox}{Exercice n°11}
On considère un repère \Oij{} du plan.

\begin{enumerate}
    \item Déterminer une équation cartésienne de la droite $\mathscr{D}$ passant par le point $A(3;1)$ et de vecteur directeur $\Coord{u}{-1}{5}$
    \item Déterminer une équation cartésienne de la droite $\mathscr{D'}$ passant par les points $B(5;3)$ et $C(1;-3)$
\end{enumerate}

\end{mybox}

\begin{framed}
\vspace{3cm}
\end{framed}


\end{document}
